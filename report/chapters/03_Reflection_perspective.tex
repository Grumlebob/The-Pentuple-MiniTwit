\section{Reflection Perspective}

\subsection{Biggest issues}

These are some of the problems we spent the most time solving.

\subsubsection{Migrations and Docker compose}

In the beginning of the project after we had refactored MiniTwit to .NET,
we ran the program with \texttt{dotnet run}. 
When the API started up it would also perform a database migration,
in case the model had changed. This is a part of EF core
that makes it easy to evolve the database schema.
However, when we started using Docker compose, it would not run the migration.
We spent a lot of time trying to make it work.
Additionally, we did not have much experience with Docker,
resulting in us spending over 30 hours total on the ticket
``Dockerize application''.

As a solution we ended up creating a separate Dockerfile for the migration.
The sole purpose of this file was to use the API code project to run the migration.
Later on as we implemented the CI/CD pipeline, 
the migration was made part of the deployment process.

\subsubsection{Docker swarm}

Before we implemented docker swarm, we had a single droplet 
and a single docker compose file that defined all the MiniTwit modules.
This made it quite the challenge to change to several droplets with 
seperate docker compose files for each module.
With a single compose file it was easy to have health checks and
dependencies between the modules. 
Fortunately, the only crucial dependency was the database migration.
The migration cannot happen before the database is up and running.
However, with the database continuously running on its own droplet,
it is no longer an issue.

Another related issue was where to run the migration.
Since the API module is the one communicating with the database,
we initially included it in the API docker compose file.
However, this was an issue now that the API was running on several 
worker droplets as part of a docker swarm. Workers are designed
to be able to crash and start up again, 
meaning it is unpredictable when they run their given services.
We want to only run the migration once, when we deploy.
Otherwise, multiple concurrent migrations could cause 
race conditions or similar issues on the database.
This was solved by separating the migration into its own docker compose file.
This is run once on the leader node, when we deploy.
We ended up spending over 14 hours on the ticket ``Set up docker swarm''.

\subsection{Lesssons learned}

Automation is worth it.
Clickops is time consuming and error prone.

with regards to
evolution and refactoring
operation, and
maintenance

\subsection{Unresolved issues}

There are some issues that we have still not managed to solve.

\subsubsection{Domain url has long load time}

For some reason when going to the client with the https domain name, 
it loads for several seconds before the page is shown.

\subsubsection{A hidden dependency}

The API must allow the client to access it by 
setting the CORS (Cross-Origin Resource Sharing) policy.
We do this in the appsettings. The problem is that it is done manually.
When the client changes IP-address, we must include 
that IP in the CORS policy of the API.
We could allow all origins to avoid this issue, 
but optimally we would have liked to dynamically 
inject it somewhere in the workflow.

\subsubsection{Seq limitations}

Adding monitoring with Seq was easy, since it is made specifically for .NET.
However, it has some limitations, regarding graphs.
It has a specific format for the SQL queries you can make graphs from.
Specifically, it can only contain the following operations.
\texttt{select, where, group by, having, order by, limit}.
We wanted to make a business relevant graph testing the 1\% rule\cite{1_perc_rule}.
In short, group users based on how many posts they have made.
In order to combine this data, we only found ways that involve 
\texttt{join} operations, which Seq does not support.

Furthormore, we have not found a way to save the graphs we make.
This means, if we tear down the droplet running Seq,
we lose all the custom graphs we have made.
To avoid starting completely over, we have saved the queries,
so they can be manually pasted back in, when we run Seq again.

\subsubsection{File structure}

Terraform was one of the last things we added to the project.
Currently, the terraform files are placed in the ``MiniTwitSoultion'' 
folder next to ``remote\_files'' which are the files copied to the droplets.
If the project were to continue, we would move these things to a deployment folder.

\subsection{The DevOps difference}

These are the things that made this project different and more devopsy
than other projects we have worked on.

\subsubsection{Continuous integration/deployment}

Deployment was a big focus in this project, 
which is something we are not very used to.
The issues we spent the most time on were often related to deployment.
However, once the workflow was in place, it was
an awesome advantage to be able to push some code,
whereafter it was automatically available for everyone on the url, 
meaning there was no need to run the program locally to see the results.

\subsubsection{Monitoring}

Monitoring is also something we haven't used much in previous projects.
But given how easy it was to add, and the advantages it gave,
we will definitely include it in future projects.
It was super useful for detecting errors, unexpected behavior,
and slow response times, as well as general traffic to see 
what kind of requests the program receives.


\subsection{Reasons for tool choices}

.NET, minimal api, Blazor,

Docker

Digital ocean.

Seq, serilog

Terraform

Csharpier

SonarQube

Postgres

dependabot

coveralls

testcontainers

github actions