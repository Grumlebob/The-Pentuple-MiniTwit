\section{Security assessment expanded}
\label{appn:B}

\subsubsection{Access to Seq (negligible, likely)}

In the case of resetting/failure of the Seq droplet, the logs 
are lost. Morever, the Seq password is reset everytime, meaning we 
do not have a fail safe default. This gives malicious users temporary 
access to our Seq dashboards, until we reset the password. However, 
this is not a big issue, as Seq is only used for monitoring and logging, 
and does not contain any sensitive information. However, only one user 
(stakeholder or developer) can access Seq at a time, 
which can be abused to deny availability to legitimate users.

\subsubsection{Login credentials leaked (moderate, likely)}

Usernames and passwords are stored in plaintext in our 
database (no encrpytion or hashing applied). If an attack 
agains read access to the users table, all credentials are 
immediately comprised, leading to unathorized account takeover. 

\subsubsection{Overloading the API with requests (moderate, possible)}

A DDoS (distributed denial of service) attack can flood our API with 
excessive traffic, rendering it unavailable to legitimate users. 
We do have load-balancing and uncomplicated firewalls, but there is no 
safe guard against DDoS attacks. Neither do we enfornce CAPTCHA or 
rate-limiting on critical endpoints. 
The impact is limited to service unavailablity. 

\subsubsection{Access to the database (catastrophic, unlikely)}

If an attack obtains direct database credentials or exploits a 
vulnerability to gain read/write access. They could delete or modify 
critical data, or have access to all user data, including usernames 
and passwords.

\subsubsection{Leak of secrets and API keys (catastrophic, possible)}

Some of the project secrets are shared informally (e.g., via Discord). 
Thus if the wrong person is invited, these keys could be exposed. 
An attacker with a valid API key could impersonate our services, 
or cause damage. The probability is possible until stricter 
secret-management policies are enforced, 
such as using a secret manager or vault.

\subsubsection{Droplet failure (catastrophic, possible)}

Any droplet could fail due to software or hardware issues. We currently have 
no database backups. If the database droplet dies, we lose all data. 
The impact is catastrophic as we could lose all data (including user accounts 
and tweets) and the application would be unavailable until the droplet is 
restored or replaced.
