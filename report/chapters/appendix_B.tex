\section{Security assessment expanded}
\label{appn:B}

This section gives a more detailed description of the security risks
of our \texttt{MiniTwit} application.

\subsection{Access to Seq (negligible, likely)}

In the case of resetting/failure of the \texttt{Seq} \texttt{droplet}, the logs 
are lost. Moreover, the \texttt{Seq} password is reset every time, meaning we 
do not have a fail-safe default. This gives malicious users temporary 
access to our \texttt{Seq} dashboards, until we reset the password. However, 
this is not a big issue, as \texttt{Seq} is only used for monitoring and logging, 
and does not contain any sensitive information. However, only one user 
(stakeholder or developer) can access \texttt{Seq} at a time, 
which can be abused to deny availability to legitimate users.

\subsection{Login credentials leaked (moderate, likely)}

Usernames and passwords are stored in plain text in our 
database (no encryption or hashing applied). If an attack 
gains read access to the users table, all credentials are 
immediately comprised, leading to unauthorized account takeover. 

\subsection{Overloading the API with requests (moderate, possible)}

A \texttt{DDoS} (distributed denial of service) attack can flood our \texttt{API} with 
excessive traffic, rendering it unavailable to legitimate users. 
We do have load-balancing and basic firewalls, but there is no 
further safeguard against \texttt{DDoS} attacks. Neither do we enforce \texttt{CAPTCHA} or 
rate-limiting on critical endpoints. 
The impact is limited to service unavailability. 

\subsection{Access to the database (catastrophic, unlikely)}

If an attack obtains direct database credentials or exploits a 
vulnerability to gain read/write access. They could delete or modify 
critical data, or have access to all user data, including usernames 
and passwords.

\subsection{Leak of secrets (catastrophic, possible)}

Some of the project secrets are shared informally (e.g., via \texttt{Discord}). 
Thus, if the wrong person is invited, these keys could be exposed. 
An attacker with a valid key could impersonate our services, 
or cause damage. The probability is possible until stricter 
secret-management policies are enforced, 
such as using a secret manager or vault.

\subsection{Droplet failure (catastrophic, possible)}
Any \texttt{droplet} could fail due to software or hardware issues. We currently have 
no database backups. If the database \texttt{droplet} dies, we lose all data. 
The impact is catastrophic as we could lose all data (including user accounts 
and tweets), and the application would be unavailable until the \texttt{droplet} is 
restored or replaced.